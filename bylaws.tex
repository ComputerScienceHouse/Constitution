% BEFORE CHANGES ARE MADE TO THIS DOCUMENT:
% -References will be automatically updated if any part is added, deleted, etc.
%  However, if a sub part is moved to a different part, its references must be
%  changed.
% -This document must be ratified by the House (as per the Constitution),
%  then printed, signed, notarized, and placed in the House filing cabinet
%  if changes are to be finalized.

\documentclass{article}
\providecommand{\RevisionInfo}{}
% The xr package allows external references
\usepackage{xr-hyper}
\usepackage{hyperref}
% Reformat section titles
\usepackage{titlesec}

% This package is useful for debugging label problems
% Comment out in final revision
%\usepackage{showkeys}

% Define the external document to be constitution for cross referencing purposes
%\externaldocument{constitution}[https://github.com/ComputerScienceHouse/Constitution/blob/master/constitution.pdf?raw=true]
\externaldocument{articles}[articles.pdf]

% Title page information
\title{Computer Science House By-Laws}
\author{Computer Science House Constitution Committee}
% Last Modified Date
\newcommand{\datechanged}{Last Updated: \RevisionInfo}
\date{\datechanged}

% Fix margins
\setlength{\evensidemargin}{0in}
\setlength{\oddsidemargin}{0in}
\setlength{\textwidth}{6.5in}
\setlength{\topmargin}{0in}
\setlength{\textheight}{8.5in}

% Use \article for articles and \asection for sections of articles.
% Automatically provide labels with the same article or section title.
\newcommand{\bylaw}[1]{\section{#1} \label{#1}}
\newcommand{\bsection}[1]{\subsection{#1} \label{#1}}
\newcommand{\bsubsection}[1]{\subsubsection{#1} \label{#1}}
\renewcommand{\thesection}{\Roman{section}}
\renewcommand{\thesubsection}{\arabic{section}.\Alph{subsection}}
\renewcommand{\thesubsubsection}{\arabic{section}.\Alph{subsection}.\arabic{subsubsection}}
\titleformat{\section}{\normalfont\Large\bfseries}{By-Law \thesection}{1em}{}
\titleformat{\subsection}{\normalfont\large\bfseries}{Section \thesubsection}{1em}{}

% Adding a \bsubsubsection -- I feel dirty
\setcounter{secnumdepth}{5}
\newcommand{\bsubsubsection}[1]{\paragraph{#1} \label{#1}}
\renewcommand{\theparagraph}{\arabic{section}.\Alph{subsection}.\arabic{subsubsection}.\Alph{paragraph}}

% Headings
\pagestyle{myheadings}
\markright{{\rm CSH Constitution \hfill \datechanged \hfill Page }}

% Reference example:
%Test reference \ref{House Objectives} House Objectives.


\begin{document}
% Title
\maketitle

\tableofcontents

% BY-LAW I - INTRODUCTION
\bylaw{Introduction}
By-Laws will expand upon, while not redefining, Articles laid out in the Constitution.
By-Laws provide additional depth and definition to the processes and operations of House.

% BY-LAW II - GENERAL OPERATIONS OF THE HOUSE
\bylaw{General Operations of the House}
\bsection{Standard Operating Session}
The Standard Operating Session for Computer Science House is during the twenty-eight weeks of Fall and Spring semesters.
The Summer Sessions, Intersessions, Institute breaks, and all end of Semester Exam Weeks are considered non-standard operating sessions.
Unless explicitly stated otherwise, the requirements and expectations defined in the constitution are for the Standard Operating Session.

% BY-LAW III - OPERATIONS OF THE FINANCIAL DIRECTORSHIP
\bylaw{Operations of the Financial Directorship}
\bsection{Amount of House Dues}
The amount of dues for Active Members will be eighty dollars (\$80.00) per Academic Semester.
\bsection{Collection of House Dues}
The collection period of house dues will be decided in conjunction with The Center for Residence Life and Housing Operations.
During the dues collection period, dues for on-floor members (for both semesters) are collected through the member’s RIT bill.
Dues for off-floor members are to be collected during this period as well by the Financial Director.
For any member who moves on or off-floor during the year, any difference between dues paid and dues owed should be collected.
Dues for any alumnus/alumnae in good standing are to be collected when intention to pay is expressed to the Financial Director.
\\*\\*
If a member is unable to pay dues upon request, they may appeal their situation to the Financial Director.
If the Financial Director deems their situation appropriate, they may grant an extension for payment.
If the Financial Director deems their situation inappropriate for extension, the member may appeal to the Executive Board.
If the Executive Board disagrees with the Financial Director, the Executive Board may grant an extension.
If the Executive Board agrees with the Financial director, the member’s payment is considered delinquent and the member’s privileges are revoked until payment can be made.
This process may be repeated indefinitely at the discretion of the Financial Director and the Executive Board.
\\*\\*
All dues must be collected in full before the annual membership evaluation (\ref{Membership Evaluation}).
After this date, dues collection is suspended until the start of the next academic year.
\bsection{Breakdown of Dues for Directorship Budgets}
\begin{center}
\begin{tabular}[c]{l c}
Directorship Name & Percentage of Dues \\
\hline
\hline
Operational Communications & 20\% \\
\hline
Evaluations \& Selections & 5\% \\
\hline
House History & 10\% \\
\hline
House Improvements & 15\% \\
\hline
Research and Development & 20\% \\
\hline
Social & 20\% \\
\hline
Accumulated & 10\% \\
\hline
\end{tabular}
\end{center}

The suggested total operational budget is \$8064.
This figure is based on yearly estimated on-floor member dues (\$160 x 72 members = \$11520) minus the 10\% reserved for Accumulated (Accum).
Money allocated for Accum and off-floor member dues is deposited into the CSH Account and saved.

\bsection{Expenditure Approval}
All House expenditures must be approved by both the Financial Director and the appropriate Director (from whose budget the funds will be drawn).
Single expenditures may not exceed seventy-five dollars (\$75) and total expenditures may not exceed one-hundred dollars (\$100).
A total expenditure is defined as all the funds drawn for a specific event, project, piece of equipment, or service.
\\* \\*
If the above amounts are to be exceeded, the expenditure must be approved by a Simple Majority vote at House meeting before the funds may be appropriated, as defined in \ref{Simple Majority}.
At the Financial Director's discretion, a Spending Committee meeting may be held in order to approve any purchase exceeding one-hundred dollars (\$100), but not to exceed three-hundred dollars (\$300).
A quorum of one-quarter the number of current active members is required for the meeting to take place and for any vote to be considered valid, it must be passed by a simple two-thirds majority.
\\*\\*
For expenditures exceeding \$300 in total funding, a Complete Project Proposal must be presented to House at a House meeting.
This proposal includes the project budget, inventory of required resources, and a timeline for completion.
\\* \\*
If an appropriate Directorship cannot be determined for an expenditure, it is to be brought up for approval at House meeting as a Miscellaneous expenditure and approved by a simple majority vote, regardless of the amount.
This amount is to be directly subtracted from the general CSH account.
\\* \\*
Funds allocated for a project not spent by the end of the Standard Operating Session \ref{Standard Operating Session} in which they were approved are voided, unless otherwise specified during the original voting process.

% BY-LAW IV - OPERATIONS OF THE EVALUATIONS DIRECTORSHIP
\bylaw{Operations of the Evaluations Directorship}
\bsection{The Introductory Process}
The Introductory Process is designed to provide an easy means for Introductory Members to meet existing House members, learn House history, demonstrate their involvement potential to House, and allow existing House members to evaluate them for Active Membership.
\bsubsection{The Evaluation Period}
The Process lasts between six (6) and ten (10) weeks, starting when the Evaluations Director initiates it.
The Executive Board may hold a closed ballot Simple Majority vote to extend the Introductory Process.
\bsubsection{The Introductory Project}
Introductory members are expected to demonstrate initiative and an ability to work well in a group environment through the Introductory Project.
\bsubsection{Introductory Executive Board}
An Introductory Executive Board is elected by the Introductory Members to lead and manage the project, represent the introductory project participants to House, and evaluate participants based on their contribution to the project.
The Introductory Executive Board consists of three members:
\begin{itemize}
	\item A President, who leads the project and acts as primary representative to House through weekly status reports at House Meeting.
	\item A Vice President, who assists the President in leadership and representation.
	\item A Treasurer, who manages the finances for the project.
\end{itemize}
Introductory Members also elect a Secretary to take notes, and any number of additional non-voting leaders may be elected to take on project-specific tasks.
\bsubsubsection{The Project}
The Project is an annual fundraising event that takes place in the first semester of the Introductory Process.
All of the proceeds go to a charity chosen in advance by the participants.
The Executive Board is encouraged to discuss the Project and following an open ballot Simple Majority Vote, may choose to intervene in the Introductory Project if the Project does not appear to fulfill requirements.
\bsubsubsection{Introductory Project Advisors}
The Evaluations Director may choose between one (1) and three (3) House members to be Introductory Project Advisors.
The advisors assist in organizing the participants at the beginning of the Introductory Project and thereafter act as a resource to keep the Project running smoothly.
\bsubsection{The Introductory Packet}
Each Introductory Process participant, after the first week, is given two (2) weeks to return a document that should contain the following:
\begin{itemize}
	\item A signature from all Active/Introductory Resident members, each Executive Board member, and fifteen (15) of any combination of the following:
	\begin{itemize}
		\item Non-Resident members who have passed a Membership Evaluation
		\item Alumni members (not including Non-Resident Executive Board members)
		\item Honorary members
		\item Advisory members
	\end{itemize}
	\item A description of each Executive Board position, including:
	\begin{itemize}
		\item The responsibilities of the position
		\item The name(s) of the position’s member(s)
		\item The position’s weekly meeting time
	\end{itemize}
	\item A list of seven (7) annual House social events
	\item A list of seven (7) major House technical achievements
\end{itemize}
\bsubsection{Expectations of an Introductory Member}
Before the end of the Process, an Introductory Member is expected to:
\begin{itemize}
\item Attend all House meetings during the process
\item Complete the Introductory Packet
\item Be an active participant in the Introductory Project
\item Attend at least one House directorship meeting for each week of the process (including permanent and Ad-Hoc directorship meetings)
\item Attend at least one House social event during the process
\item Attend at least two seminars relating to computing or electronics
\item Sign a copy of the Membership Agreement describing a House Member’s responsibilities and expectations
\end{itemize}

\bsubsection{Introductory Evaluation}
The Introductory Evaluation is the process by which House chooses which Introductory Members to extend an offer of full Resident or Non-Resident Membership.
Occurs once per academic year as part of the Introductory Process.
\bsubsubsection{Voting}
On a member by member basis, this evaluation process determines if each Introductory member has successfully completed the Introductory Process requirements \ref{Expectations of an Introductory Member}.
A Simple Majority vote with a quorum of two-thirds of the Total Number of Possible Votes is required for the evaluation to be valid.
Neither absentee nor proxy votes are allowed.
House may choose any of the Outcomes for each member.
\bsubsubsection{Outcomes}
\renewcommand{\theenumi}{\alph{enumi}} % For this section, we want items to use letters
\begin{enumerate}
	\item Introductory Members may be offered Active Membership (\ref{Active Membership}) provided they meet the requirements described in \ref{Expectations of an Introductory Member}, at the discretion of House.
	\item If an Introductory Member fails to meet the requirements, their membership will be revoked.
		In addition, the participant is asked to find alternative housing as soon as possible in accordance with all applicable Residence Life policies regarding room changes.
	\item An Introductory member may be given a conditional to complete as a means of making up for missing requirements.
		A conditional may be proposed by any member present at the Introductory Evaluation and, if it is approved by the Evaluations Director, is then voted on by House.
		Each conditional consists of a set of additional requirements a deadline for completing them.
		When the deadline expires, the conditional will be brought before the Executive Board who will assess its completeness.
		A conditional member who does not meet these additional requirements forfeits membership.
\end{enumerate}
\bsection{Selection Processes}
\renewcommand{\theenumi}{\alph{enumi}} % For this section, we want items to use letters
\bsubsection{Selection Process for Students Attending RIT for at Least One Semester}
\begin{enumerate}
	\item The applicant submits an application to the Evaluations Director for review.
	\item The applicant participates in an informal interview with exactly three current Active, Alumni, or Honorary Members.
	\item The application materials are then reviewed at an Evaluations meeting.
	Then a simple majority vote, of attending members, is held on whether or not to accept the person as an Introductory Member.
\end{enumerate}
Note: Most membership privileges do not initiate until successful completion of the introductory process.
This means that until the member has passed the introductory evaluation, the member does NOT have the right to vote on House issues and does NOT count towards quorum.
The member does, however, have the right to use Computer Science House's facilities.
The member must also pay dues.
\\* \\*
Additional Note: No hazing shall occur at any time during the Selection Process in accordance with the New York State Hazing Laws.
\bsubsection{Selection Process for First Semester and Entering RIT Students}
\begin{enumerate}
	\item During Spring semester, the Evaluations Director selects a group of Active Members to form a Selections Committee.
		The Selections Committee reviews applications and conducts interviews in accordance with the process prescribed by Residence Life.
	\item A subset of applications will be selected to move onto the House Fall Semester, are granted full membership privileges (including On-Floor status), and must participate in the Introductory Process as defined in \ref{The Introductory Process}.
		An additional subset will be given Off-Floor status, but otherwise receive the same privileges and responsibilities.
\end{enumerate}
Note: Most membership privileges do not initiate until successful completion of the introductory process.
This means that until the member has passed the introductory evaluation, the member does NOT have the right to vote on House issues and does NOT count towards quorum.
The member does, however, have the right to use Computer Science House's facilities.
The member must also pay dues.
\\* \\*
Additional Note: No hazing shall occur at any time during the Selection Process in accordance with the New York State Hazing Laws.
\bsubsection{Selection Process for Honorary Members}
\begin{enumerate}
	\item A House Member submits to the Evaluations Director a nomination for a person they feel is deserving of Honorary Membership.
	\item The Evaluations Director performs some preliminary research on the candidate and presents the findings.
	\item The Executive Board decides whether or not to present the nomination to the House for a secret ballot House vote.
		If the Executive Board decides not to present the nomination to the House, the Selection Process ends and the candidate does not become an Honorary Member.
	\item The nomination is presented at a House Meeting for discussion and a Two-Thirds vote, as described in \ref{Two-Thirds}.
		Ballots are distributed and voting must remain open for a minimum of forty-eight hour period.
	\item The candidate is notified of their selection as an Honorary Member and presented with the honor.
\end{enumerate}
\bsubsection{Selection Process for Advisory Members}
\begin{enumerate}
	\item When there is a perceived need, the Executive Board may open nominations for Advisory Members.
		After an announcement at House Meeting, during a 72-hour period, any House Member may submit a nomination.
	\item After the close of the nomination collection period, the Executive Board will arrange some means for the House to meet with the nominees.
	\item A discussion of the candidates will be held at the following House meeting.
	\item Ballots are distributed for each nominee as a fifty-percent House vote defined in \ref{Fifty Percent}.
		Voting must remain open for a minimum of a forty-eight hour period.
	\item All candidates selected are notified of their acceptance as House Advisors and asked to accept or decline the selection.
\end{enumerate}

\bsection{Evaluations Processes}
During the academic year, a House member is evaluated by the Membership Evaluation.
The Membership Evaluation is responsible for determining those individuals who will have the option to continue Active Membership in the following year.
Semi-Annual Evaluations Meetings are open only to current Active Members.
All Executive Board members are expected to attend the Semi-Annual Evaluations Meetings to assist in the evaluations of House members.
\\* \\*
At the beginning of any Evaluation Process listed herein, the Evaluations Director must read the sections of the Computer Science House Constitution and By-Laws used during the respective Evaluation Process.
It is incumbent upon each House member to provide the Evaluation Director with whatever information they feel is necessary to ensure an accurate evaluation.
\bsubsection{Membership Evaluation}
The Membership Evaluation process occurs once per academic year.
It is performed as part of the Evaluation Process that takes place during the spring semester to comply with RIT Housing deadlines.
\bsubsubsection{Voting}
A quorum of members, on a member by member basis, ensures that each Active Member has met the minimum requirements in order to continue as an Active Member in the following year.
All Active Members will be held to the requirements and evaluated unless they have received an exemption from the Executive Board prior to the event.
These requirements are detailed in Article V under expectations for each category of membership.
These requirements must be completed before the Membership Evaluation occurs.
A secret Simple Majority vote with a quorum of two-thirds of the Total Number of Possible Votes is required for the evaluation to be valid.
Neither absentee nor proxy votes are allowed.
House may choose any of the Outcomes for each member.
\bsubsubsection{Outcomes}
Members who pass Membership Evaluation have the option to participate as an Active Member in the following year and remain on the upcoming roster if applicable.
All members who pass Membership Evaluation will recieve two housing points for use in the following academic years.
\\* \\*
If a member fails and has never passed a Membership Evaluation in the past, their membership will be revoked immediately.
If the member has previously passed a Membership Evaluation they will move to Alumni Membership status at the end of the Standard Operating Session (\ref{Alumni Membership Selection}).
In either case, the member forfeits their ability to be included on the following year’s roster.
\\* \\*
A member may be given a conditional to complete as a means of making up for missing requirements.
A conditional may be proposed by any member present at the Evaluation.
If the conditional is approved by the Evaluations Director, it is then voted on by House.
Each conditional consists of a set of additional requirements a deadline for completing them.
When the deadline expires, the conditional will be brought before the Executive Board who will assess its completeness.
A conditional member who does not meet these additional requirements will have failed the evaluation.
\bsubsection{Appeals Process}
If a member disagrees with the outcome of any evaluation, (e.g. is not asked to return to the House for the following year) and wishes to appeal the decision, they may do so as stated in \ref{Appeals}.
\\* \\*
If the member is still unsatisfied after being heard by the Executive Board, the appeal may be brought to the attention of the ResLife Advisor.

\bsection{Expectations of House Members}
Each member is required to pay House dues as stated in \ref{Collection of House Dues}, attend all House Meetings, and attend at least fifteen (15) of the House directorship meetings (including permanent and Ad-Hoc directorship meetings) for each Academic Semester in which they are an Active Member.
\\* \\*
The member must participate significantly in at least one major project during the academic year.
The member is required to submit a description for this major project to the Executive Board for approval.
As an option to this requirement, the member may instead assist on a large number of House activities and projects.
However, it is to be understood in advance by the member that this option requires a great deal of participation throughout the year.
This participation will be evaluated by the Executive Board.

\bsection{Housing Status}
All Active and Introductory Members have a housing status.
This status indicates their right to priority housing on the floor.
Any Alumni Member in good standing will maintain their previous On Floor status upon a return to Active membership.
\bsubsection{On-Floor Status}
Members with On-Floor status are eligible to live on the floor.
To be granted On-Floor status, members may notify the Evaluations Director that they would like to come up for a vote.
The Evaluations Director will then bring them up for vote at the next meeting.
\bsubsection{Off-Floor Status}
Members with Off-Floor status do not have the right to live on the floor.
They still have access to all other privileges associated with their membership, and may still accumulate Housing Points.

\bsection{Housing Priority System}
The Housing Priority System is a means for determining the priority a House member has in a Housing issue such as Single Room Assignment or the Assignment of Available Housing.
The member with top priority is the member with on-floor status and the most Housing Priority Points.
Housing points are accumulated during the Membership Evaluation described in \ref{Membership Evaluation}.
\\* \\*
In the event of a tie, the members will be approached simultaneously and if they are unable to decide fairly between themselves, the assignment of priority will be made by random selection of the tied members.
\bsubsection{Single Room Assignments}
When a single room on the House becomes available it is offered to the member who carries the highest Housing Priority as defined in By-Law V, Section H.
If that House member declines the option, it will be offered to the member with the next highest Housing Priority.
This process continues until a member selects to move into the single room.
Once in a single room, a House member retains the assignment until voluntarily giving it up.
It should be noted that members selecting this option must agree to any additional charges applied by the Department of Residence Life for residing in a single room.
\bsubsection{Double Rooms as Single rooms}
During the third week of each semester, if there is no waiting list for residency on the House, any vacant rooms will be offered to House members as single rooms assignments according to the method as described in \ref{Single Room Assignments}.
It should be noted that this does not mean members will be relocated into empty spaces so that the member with top priority is offered the single.
This only applies if there is a totally vacant room.

% BY-LAW V - OPERATIONS OF THE OPERATIONAL COMMUNICATIONS DIRECTORSHIP
\bylaw{Operations of the Operational Communications Directorship}
The Operational Communications Directorship is responsible for overseeing the implementation of maintenance and upgrades to the House Computer Systems Network.
It is a self-governing committee making decisions by a simply majority vote.
Membership is composed of all Root Type Persons.

\bsection{Selection of a Root Type Person}
\renewcommand{\theenumi}{\alph{enumi}} % For this section, we want items to use letters
\begin{enumerate}
	\item Nominations are taken from the Root Type Persons and Prior Root Type Persons meeting the selection criteria in \ref{Qualifications of a Root Type Person}.
	\item Each candidate is given a minimum of twenty-four hour period to accept or decline the nomination.
	\item A list of all nominees who have accepted is presented to the Executive Board for approval.
		This Executive Board Meeting is closed to the Executive Board Members, Root Type Persons, and House members with explicit invitation from the Executive Board.
	\item If an Executive Board member is a candidate for the office in discussion, the member is absent and their vote is abstained.
		A simple majority Executive Board vote, as described in \ref{Simple Majority}, is taken to determine whether the nominations of the Root Type Persons are accepted.
\end{enumerate}

\bsection{Qualifications of a Root Type Person}
\renewcommand{\theenumi}{\alph{enumi}} % For this section, we want items to use letters
\begin{enumerate}
	\item Candidates must have been an On-Floor Active Member in good standing within the last twelve months, or been an Off-Floor Active Member and have lived on the floor within the last twelve months.
	\item Candidates may be granted an exemption by current Root Type Persons.
		Such an exemption may be withdrawn at any time by the current Root Type Persons.
	\item Prior Root Type Persons are those members who are no longer current Active Members and have not been granted an extension by the current Root Type Persons.
		Prior Root Type Persons are not guaranteed access to the current root passwords and other authentication tokens.
	\item The current Root Type Persons may from time to time draft rules and regulations specifying the rights and privileges of Prior Root Type Persons.
\end{enumerate}

\bsection{Creation of Accounts}
Root Type Persons have the authority to manage user accounts for House systems.
Before members may receive an account they must:
\renewcommand{\theenumi}{\alph{enumi}} % For this section, we want items to use letters
\begin{enumerate}
	\item Sign the Code of Conduct sheets pertaining to the responsible utilization of Computer Science House and Rochester Institute of Technology facilities.
	\item Obtain greater than or equal to 60\% (rounded up to the nearest whole person) of required signatures (excluding those of Resident members who have not passed a Membership Evaluation) in the Introductory Packet or successfully complete Introductory Evaluations.
\end{enumerate}

\bsection{Code of Conduct}
The Code of Conduct located at \url{https://github.com/ComputerScienceHouse/CodeOfConduct} is the canonical Code of Conduct for Computer Science House accounts.
\\* \\*
Members are bound to the Code of Conduct revision that they sign when initially creating their account.
Members may sign a more recent revision of the Code of Conduct to update their agreement.
\bsubsection{Changes}
The following process is used for making changes to the Code of Conduct document.
\begin{enumerate}
	\item A Root Type Person drafts a change to the Code of Conduct and makes publicly available both the summary and difference file of the change.
	\item The change is proposed at a House Meeting and is discussed at the same House Meeting.
	\item The final proposal is presented at the next House Meeting and ballots are distributed for a Balloted House vote as described in \ref{Balloted Vote}.
\end{enumerate}
\end{document}
