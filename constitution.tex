% BEFORE CHANGES ARE MADE TO THIS DOCUMENT:
% -References will be automatically updated if any part is added, deleted, etc.
%  However, if a sub part is moved to a different part, its references must be
%  changed.
% -This document must be ratified by the House (as per the Constitution) if
%  changes are to be officialized.

\documentclass{article}
\providecommand{\RevisionInfo}{}
\usepackage{hyperref}

% Reformat section titles
\usepackage{titlesec}

% This package is useful for debugging label problems
% Comment out in final revision
%\usepackage{showkeys}

% Title page information
\title{Computer Science House Constitution}
\author{Computer Science House Constitution Committee}

% Last Modified Date
\newcommand{\datechanged}{Last Updated: \RevisionInfo}
\date{\datechanged}

% Fix margins
\setlength{\evensidemargin}{0in}
\setlength{\oddsidemargin}{0in}
\setlength{\textwidth}{6.5in}
\setlength{\topmargin}{0in}
\setlength{\textheight}{8.5in}

% Use \article for articles and \asection for sections of articles.
% Automatically provide labels with the same article or section title.
\newcommand{\article}[1]{\section{#1} \label{#1}}
\newcommand{\asection}[1]{\subsection{#1} \label{#1}}
\newcommand{\asubsection}[1]{\subsubsection{#1} \label{#1}}
\renewcommand{\thesection}{\Roman{section}}
\renewcommand{\thesubsection}{\arabic{section}.\Alph{subsection}}
\renewcommand{\thesubsubsection}{\arabic{section}.\Alph{subsection}.\arabic{subsubsection}}
\titleformat{\section}{\normalfont\Large\bfseries}{Article \thesection}{1em}{}
\titleformat{\subsection}{\normalfont\large\bfseries}{Section \thesubsection}{1em}{}

% Adding an \asubsubsection -- I feel dirty
%\setcounter{secnumdepth}{5}
\newcommand{\asubsubsection}[1]{\paragraph{#1} \label{#1}}
\renewcommand{\theparagraph}{\arabic{section}.\Alph{subsection}.\arabic{subsubsection}.\Alph{paragraph}}

% Adding \a(sub){3,4}section during merge of bylaws and articles -- I feel _really_ dirty
\setcounter{secnumdepth}{7}
\setcounter{tocdepth}{7}
\newcommand{\asubsubsubsection}[1]{\parindent=0em\subparagraph{#1} \label{#1}}
\renewcommand{\thesubparagraph}{\arabic{section}.\Alph{subsection}.\arabic{subsubsection}.\Alph{paragraph}.\arabic{subparagraph}}

\newcounter{asubsubsubsubsection}[subparagraph]
\renewcommand{\theasubsubsubsubsection}{\arabic{section}.\Alph{subsection}.\arabic{subsubsection}.\Alph{paragraph}.\arabic{subparagraph}.\Alph{asubsubsubsubsection}}
\newcommand{\asubsubsubsubsection}[1]{\parindent=0em\refstepcounter{asubsubsubsubsection}\par\textbf{\theasubsubsubsubsection\hspace{1em}#1 \label{#1}}}

% Headings
\pagestyle{myheadings}
\markright{{\rm CSH Constitution \hfill \datechanged \hfill Page }}

\begin{document}
% Title
\maketitle

\tableofcontents
\setcounter{page}{4}

% ARTICLE I - ABBREVIATIONS
\article{Abbreviations and Definitions}

\subsection{Abbreviations}
% Formatted as abbreviation  full term
\begin{tabular}{ l  l  l l l l}
	Chair   & \textit{Chairperson}            & Maintainer(s) & \textit{Constitutional Maintainer(s)} \\
	CSH     & \textit{Computer Science House} & OpComm        & \textit{Operational Communications}   \\
	E-Board & \textit{Executive Board}        & PR            & \textit{Public Relations}             \\
	Eval(s) & \textit{Evaluation(s)}          & R\&D          & \textit{Research and Development}     \\
	History & \textit{House History}          & ResLife       & \textit{Residence Life}               \\
	IMPs    & \textit{House Improvements}     & RTP(s)        & \textit{Root Type Person(s)}          \\
	Intro   & \textit{Introductory}           & Social        & \textit{Socials}                      \\                                  \\
\end{tabular}

\subsection{Definitions}

% Formatted as term | definition
\begin{tabular}{l l}
	Technical Seminar          & \textit{Seminar relating to computing or electronics}                                     \\
	Standard Operating Session & \textit{RIT's fall and spring semesters, excluding institute breaks and final exam weeks} \\
	Total Expenditure          & \textit{Funds drawn for a specific event, project, piece of equipment, or service}
\end{tabular}

% ARTICLE II - INTRODUCTION
\article{Introduction}

\asection{Name}
The name of this Special Interest House is Computer Science House.

\asection{Derivation of Authority}
CSH shall recognize that it receives its right to function as a Special Interest House from the Center for Campus Life and the Department of Residence Life.

\asection{House Objectives}
The objectives of CSH are:
\begin{enumerate}
	\item To enhance the education experience of its members
	\item To offer students educational programming with an emphasis on computers
	\item To provide a variety of services for its members, the RIT campus, and the Rochester Community
	\item To provide a friendly and comfortable living environment in the residence halls
\end{enumerate}

% ARTICLE III - CONSTITUTIONAL STRUCTURE
\article{Constitutional Structure and Amendment}

\asection{Constitution}
The CSH Constitution is written and maintained by members and defines the major aspects, goals, and governing structure of CSH.
\\*\\
Unless explicitly stated otherwise, the requirements and expectations defined in the Constitution are for the Standard Operating Session.

\subsection{Modifying the Constitution}
\asubsection{Non-Semantic Changes}
There are two methods for non-semantic change to the Constitution.
A Maintainer may approve any proposed change that does not affect the meaning of the document.
Alternatively, the change may be presented at a House Meeting for discussion followed by an Immediate One-Half Vote with fifty percent quorum.

\asubsection{Semantic Changes}
Any semantic change to the Constitution requires the change to be proposed in writing for discussion at a House Meeting.
Any modifications made due to the discussion are added to the written proposal and the modified proposal is posted during the week.
The final proposal is presented the following week, and ballots are distributed for a Balloted Two-Thirds Vote with two-thirds quorum as described in \ref{Balloted Vote}.
The ballots are collected for a minimum of a forty-eight hour period.
A quorum of two-thirds of Eligible Votes must cast ballots for the vote to be official.
A vote equaling or exceeding two-thirds of the number of Votes Cast is required for the change to be placed into the Constitution.
The Constitution may be overridden by an Immediate Nine-Tenths Vote with eighty-five percent quorum as described in \ref{Immediate Vote}.

% ARTICLE IV - MEMBERSHIP
\article{Membership}
There are five types of membership available to CSH.
When describing membership types, the following terms are used:
\begin{description}
	\item[Qualifications:] What criteria an applicant needs to be eligible for membership
	\item[Selection:] The process by which an applicant gains membership
	\item[Expectations:] The duties and responsibilities associated with CSH membership
	\item[Privileges:] The benefits offered to members
	\item[Evaluations:] The process by which a member is reviewed and assessed
	\item[Leave of Absence:] The process by which a member may leave CSH during an RIT Leave of Absence
	\item[Resignations:] The process by which a member terminates CSH membership
	\item[Term:] The duration of the membership
\end{description}

\asection{Introductory Membership}

\asubsection{Introductory Membership Qualifications}
Intro Membership is open to all students currently enrolled at RIT.

\asubsection{Introductory Membership Selection}
Applicants notify the Evals Director of their interest in membership by submitting an application.
They then undergo the selection process as defined below.

\asubsubsection{Selection Process for Current RIT Students}
\renewcommand{\theenumi}{\arabic{enumi}} % For this section, we want items to use letters
\begin{enumerate}
	\item The applicant submits an application to the Evals Director for review.
	\item The applicant participates in an informal interview with three current Active, Alumni, or Honorary Members.
	\item The application materials are reviewed at an Evals meeting.
	      Then an Immediate Relative Majority Vote of attending members is held on whether or not to accept the person as an Intro Member.
\end{enumerate}

% Incoming RIT students is defined as students who have attended RIT for less than 1 full semester
\asubsubsection{Selection Process for Incoming RIT Students}
\begin{enumerate}
	\item During spring semester, the Evals Director selects a group of Active Members to form a Selections Committee.
	      The Selections Committee reviews applications and conducts interviews in accordance with the process prescribed by ResLife.
	\item A subset of applicants will be offered Intro Membership and have the opportunity to participate in the Intro Process defined in \ref{The Evaluation Period}.
	      A subset of Intro Members will be offered On-Floor status and will be able to select a room on floor.
	      The other Intro Members will have Off-Floor status, but will otherwise have the same privileges and responsibilities.
\end{enumerate}

\asubsection{Introductory Membership Expectations}
Intro Members are expected to meet all the requirements of the Intro Process described in \ref{Expectations of an Introductory Member}.
Intro Members are not required to pay membership dues.

\asubsection{Introductory Membership Privileges}
Most membership privileges do not initiate until successful completion of the Intro Process.
This means that until the member has passed the Intro Eval, the member does not have the right to vote on CSH issues and does not count towards quorum.
The member does, however, have the privilege to use CSH's facilities.

\asubsection{Introductory Membership Evaluations}
The Intro Process allows Intro Members to learn about CSH and demonstrate their interest in becoming Active Members. During this period, Intro Members will meet members, learn CSH history, and be evaluated for Active Membership by Active Members.

% fix numbers to be only text
\asubsubsection{The Evaluation Period}
The Intro Process will occur either once or twice per semester, and will last for six weeks.
The process will be initiated by the Evals Director during either the first or second week of each semester. E-Board may approve an extension to the Intro Process by an E-Board Vote.
If an Intro Process is extended, the Intro Eval shall occur at the termination of the extended process, which may be outside of the period defined in \ref{Introductory Evaluation}.
If deemed necessary by the Evals Director or by an E-Board Vote, the second Intro Process must begin within nine days of the first process ending.
If a second Intro Process is approved, the Evals Director may initiate a new Intro Packet within the first or second week of the process.

\asubsubsection{The Introductory Packet}
Each Intro Process participant, after the first week, is given two weeks to obtain signatures from all Active Members who have passed a Membership Eval, all On-Floor Intro and Active Members, and ten of any combination of Alumni, Honorary, and Advisory Members.
At the discretion of the Evals Director or as the result of an E-Board Vote, any Intro Packet may be extended to accommodate extenuating circumstances.

\asubsubsection{Expectations of an Introductory Member}
\renewcommand{\theenumi}{\arabic{enumi}} % For this section, we want items to use numbers
Before the end of the Intro Process, an Intro Member is expected to:
\begin{enumerate}
	\item Attend all House Meetings during the Intro Process
	\item Complete the Intro Packet
	\item Attend at least one directorship meeting for each week of the process
	\item Attend at least one CSH social event during the process
	\item Attend at least two Technical Seminars during the process
\end{enumerate}

\asubsubsection{Introductory Evaluation}
The Intro Eval is the process by which Active Members choose which Intro Members to extend an offer of Active Membership.
It occurs at the end of the final week of the Intro Process.

\asubsubsubsection{Voting}
On a member-by-member basis, the Intro Eval determines if each Intro Member has successfully completed the Intro Process requirements described in \ref{Expectations of an Introductory Member}.
An Immediate Relative Majority Vote with two-thirds quorum is required for the Eval to be valid.
For an Active Member to be eligible to vote during the Intro Eval they must meet the requirements outlined in \ref{Expectations of Voting Members}.
Neither absentee nor proxy votes are allowed.
Intro Members may be offered any of the outcomes listed in \ref{Outcomes}.

\asubsubsubsection{Outcomes}
\renewcommand{\theenumi}{\alph{enumi}} % For this section, we want items to use letters
\begin{enumerate}
	\item Intro Members may be offered Active Membership provided they meet the requirements described in \ref{Expectations of an Introductory Member}, at the discretion of attending members.
	\item If an Intro Member fails to meet the requirements, their membership will be revoked.
	\item An Intro Member may be given a conditional to complete as a means of making up for missing requirements.
	      A conditional may be proposed by any member present at the Intro Eval, and if it is approved by the Evals Director, is then voted on by the attending members.
\end{enumerate}

\asubsection{Introductory Membership Leave of Absence}
An Intro Member may request a Leave of Absence through the process described in \ref{Leave of Absence}.

\asubsection{Introductory Membership Resignations}
Intro Members may resign by submitting a request for termination of membership to the Chair or the Evals Director in writing before the completion of the Intro Process.

\asubsection{Introductory Membership Term}
Intro Membership shall last until the end of the Intro Process, at which time Active Membership is granted or membership is revoked.

\asection{Active Membership}

\asubsection{Active Membership Qualifications}
Active Membership is open to all students currently enrolled at RIT who have passed the Intro Eval and their most recent Membership Eval.

\asubsection{Active Membership Selection}
Alumni Members in good standing may self-select into Active Membership by paying dues to the Financial Director and notifying the Evals Director. They will retain their previous Housing Status if applicable.
\\*\\*
Any Alumni Member in bad standing, as described in \ref{Alumni Membership Selection}, may become an Active Member by notifying the Evals Director of their intent to participate.
The Evals Director will then bring them up for an Immediate Relative Majority vote with fifty percent quorum at the next House Meeting.
If the vote passes, the Alumni is reinstated as an Active Member with Off-Floor Status effective immediately.

\asubsection{Active Membership Expectations}
Active Members are expected to be active participants in CSH as defined in \ref{Expectations of Active Members}.
Active Voting Members are expected to meet the requirements defined in \ref{Expectations of Voting Members}.

\asubsubsection{Expectations of Active Members}
Active Members are required to pay dues as stated in \ref{Collection of Dues}, attend all House Meetings and E-Board candidate speeches, and attend at least fifteen of the directorship meetings for each semester in which they are an Active Member.
They are also expected to sign a copy of the Membership Agreement.
\\* \\*
Active Members must participate in at least one major project during the academic year.
Members are required to submit a description for this major project to E-Board for approval.
As an alternative to this requirement, members may instead assist on a large number of CSH activities and projects.
However, it is to be understood in advance by the member that this option requires a great deal of participation throughout the year.
This participation will be evaluated by E-Board.

\asubsection{Active Membership Privileges}
Active Members receive the privilege to:
\begin{enumerate}
	\item Vote on CSH issues
	\item Hold a position on E-Board
	\item Use CSH facilities
	\item Receive priority for on-floor housing, or apply for On-Floor Status
\end{enumerate}

\asubsubsection{Expectations of Voting Members}
\renewcommand{\theenumi}{\arabic{enumi}} % For this section, we want items to use letters

The following requirements must be met by the beginning of the Intro Eval for an Active Member to be allowed to vote.
Any of these requirements may be waived by the Evals Director or an E-Board Vote.
\begin{enumerate}
	\item Attend all House Meetings during the Intro Process
	\item Attend at least one directorship meeting for each week of the Intro Process % The number of directorship meetings intended here is equal to the number of the weeks in the process, not just one each week
	\item Attend at least one CSH social event during the Intro Process
	\item Attend at least two Technical Seminars during the Intro Process
\end{enumerate}

\asubsection{Active Membership Evaluations}
Active Members are evaluated annually through the Eval Process described below.

% ACTIVE MEMBERSHIP EVALUATIONS
\asubsubsection{Membership Evaluation}
The Membership Eval occurs once per academic year, where Active Members are evaluated.
It is performed as part of the Eval Process that takes place during the spring semester to comply with the RIT Housing deadlines.
The Membership Eval is responsible for determining individuals who will have the option to continue Active Membership in the following academic year.
This meeting is open only to Active Members.
All Active Members are expected to attend this meeting.

\asubsubsubsection{Voting}
Any Active Member who has completed all of the requirements as defined in \ref{Expectations of Active Members} at the beginning of the Membership Eval passes their Membership Eval without being voted on.
All Active Members who have not received an exemption from E-Board prior to the Membership Eval will be evaluated on a member-by-member basis by Active Members.
The Membership Eval shall hold members to the objective requirements defined in \ref{Expectations of Active Members} and determine which members may continue as an Active Member in the following academic year.
Exceptions to the requirements may be made, as attending members may choose any of the outcomes for each member, even if the member being evaluated has not completed all of their requirements.
These requirements must be completed before the Membership Eval occurs.
A Secret Immediate Relative Majority vote with two-thirds quorum is required for the evaluation to be valid.
Neither absentee nor proxy votes are allowed.

\asubsubsubsection{Outcomes}
Members who pass the Membership Eval have the option to participate as an Active Member in the following year.
\\* \\*
If a member fails and has never passed a Membership Eval in the past, their membership will be revoked immediately.
If the member fails, but has previously passed a Membership Eval, they will move to Alumni Membership in bad standing, as defined in \ref{Alumni Membership Selection}, at the end of the Standard Operating Session.
In either case, the member forfeits their ability to participate as an Active Member in the following year.
\\* \\*
A member may be given a conditional to complete as a means of making up for missing requirements.
A conditional may be proposed by any member present at the Eval.
If the conditional is approved by the Evals Director, it is then voted on by the attending members as described in \ref{Voting}.
Each conditional consists of a set of additional requirements and a deadline for completing them.
Members who are given a conditional have their Eval decision deferred, and therefore maintain their current membership status until the conditional is resolved.
When the deadline expires, the conditional will be brought before E-Board, who will assess its completeness.
A conditional member who does not meet these additional requirements will have failed the Eval.

\asubsection{Active Membership Leave of Absence}
\renewcommand{\theenumi}{\alph{enumi}} % For this section, we want items to use letters

An Active Member may at any time request a Leave of Absence using the process described in \ref{Leave of Absence}.
For the duration of an absence, a member:
\begin{enumerate}
	\item Forfeits their right to vote
	\item Does not count towards quorum
	\item Will not be required to attend House Meeting
\end{enumerate}

\asubsection{Active Membership Resignations}
An Active Member may resign by submitting, in writing, the reason for resignation to the Chair or the Evals Director.
Instead of forfeiting membership, Active Members who resign may elect to become Alumni as described in \ref{Alumni Membership Selection}.
The resignation will take effect immediately and an announcement will be made at the following House Meeting.

\asubsection{Active Membership Term}
Active Membership shall last until the member resigns or changes membership status.

\asection{Alumni Membership}

\asubsection{Alumni Membership Qualifications}
Alumni Membership is open to all former Active Members who passed at least one Membership Eval and departed for reasons other than revocation of membership.

\asubsection{Alumni Membership Selection}
Active Members who resign from CSH after passing the current operating session's Membership Eval are considered to be Alumni in good standing.
\\*\\*
Active Members who depart house without passing the current operating session's Membership Eval are considered to be Alumni in bad standing.
This may be appealed to E-Board in order to pursue a different outcome, described in \ref{Appeals}.

\asubsection{Alumni Membership Expectations}
There are no expectations associated with the Alumni Membership status.

\asubsection{Alumni Membership Privileges}
Alumni Members have the privilege to use CSH facilities and attend CSH functions.

\asubsection{Alumni Membership Evaluations}
Alumni Members are not subject to any Evals.

\asubsection{Alumni Membership Resignations}
There are no resignations associated with Alumni Membership status.

\asubsection{Alumni Membership Term}
Alumni Membership shall last indefinitely or until the member chooses to pursue Active Membership.

\asection{Honorary and Advisory Memberships}

\asubsection{Honorary Membership Qualifications}
Honorary Membership is open to a person whom CSH feels has contributed great personal effort to CSH and is deserving of recognition.

\asubsection{Advisory Membership Qualifications}
Advisory Membership is open to all members of the RIT professional, academic, or administrative staff.

\asubsection{Honorary and Advisory Membership Selection}
\begin{enumerate}
	\item A CSH member submits a nomination to the Evals Director, in writing, for a person they feel is deserving of Honorary or Advisory Membership. The nomination must specify the type of membership the nominee is nominated for.
	\item The Evals Director performs preliminary research on the candidate and presents the findings.
	\item If E-Board decides not to present the nomination at a House Meeting, the selection process ends and the candidate does not become a member.
	\item If the nomination is presented at a House Meeting for discussion. A Two-Thirds Vote with two-thirds quorum is held to determine whether the nomination is accepted.
	      Ballots are distributed and voting must remain open for a minimum of a forty-eight hour period.
	\item Candidates selected for Honorary Membership are notified of their selection as an Honorary Member and presented with the honor.
	\item Candidates selected for Advisory Membership are notified of their acceptance as CSH Advisors and asked to accept or decline the selection.
\end{enumerate}

\asubsection{Honorary Membership Expectations}
There are no expectations associated with the Honorary Membership status.

\asubsection{Advisory Membership Expectations}
There are no expectations associated with the Advisory Membership status.

\asubsection{Honorary and Advisory Membership Privileges}
Honorary and Advisory Members may advise in CSH issues, use CSH facilities, and attend CSH functions.

\asubsection{Honorary and Advisory Membership Evaluations}
Honorary and Advisory Members are not subject to any Evals.

\asubsection{Honorary and Advisory Membership Resignations}
An Honorary or Advisory Member may resign by submitting in writing the reason for resignation to the Chair.
The resignation will be read at the following House Meeting and become effective at that time.

\asubsection{Honorary and Advisory Membership Term}
Honorary and Advisory Memberships shall last until the member resigns.

\asection{Leave of Absence}
A leave of absence offers the option for members to take extended time away from their responsibilities to House for any number of personal issues including, but not limited to, physical illness, mental illness, or care giving for a sick family member.
CSH recognizes the need for its members to take care of their personal well-being and will support them through the process of requesting a leave of absence.
House will also help reacclimate any returning member back to the culture of House upon their return.
Upon approval of a leave of absence request, the Evals Director is notified and the member's leave is applied immediately.
A member is also allowed to be physically present and also be on a leave of absence.

\asubsection{Leave of Absence Request}
A leave of absence request must include the following information about the member: name, email, phone number, start date, expected return date, and a reason for the leave.
The request is then given to a staff member of ResLife, excluding student employees, for approval.
A leave of absence start date can be backdated.
At no point does E-Board need to be informed of any details relating to the reason for the leave.

\asubsection{Leave of Absence Duration}
A leave of absence can be granted for up to the length of one semester.

\asubsection{Leave of Absence Extension}
A leave of absence can be extended by submitting a new request.

\asubsection{Leave of Absence Return}
A member may return from a leave of absence whichever they choose to.
When a member wishes to return, they must in writing notify a staff member of ResLife, excluding student employees, to end their leave of absence.

\asubsection{Modified Evaluations}
A member may request their evaluation period to be extended by the duration of the leave or to end of the Standard Operating Session, whichever comes first.

\asubsection{Leave of Absence for Executive Board Member}
The duties and responsibilities of an E-Board member on leave are assumed using the process defined in \ref{Appointment of an Interim Director} until the member returns from their leave.
If the member returns before the end of the Standard Operation Session in which they started their leave, they may assume their position and the interim director is removed. % someone gets their position back if they come back during the year for which they were elected
If any point the member's leave is determined by E-Board to be detrimental to be operations of House, any E-Board member may propose an E-Board Vote to require a resignation from the member.

% ARTICLE V - OFFICERS OF THE HOUSE
\article{Computer Science House Officers}

\asection{Executive Board}
E-Board is the main governing body of CSH.
Its purpose is to provide leadership and direction for CSH, to oversee the day-to-day operations of CSH, and to initiate and organize programs and projects for CSH.
\\*\\*
There is one permanent directorship for each major aspect of the government of CSH and each one is chaired by an E-Board Member.
A Voting E-Board Member is considered to be any E-Board Member with a non-zero vote. % Defined this way so that (for example) each member in a dual directorship is each considered to be a voting member on E-Board.

\asubsection{Responsibilities}
\renewcommand{\theenumi}{\alph{enumi}} % For this section, we want items to use letters

\asubsubsection{Responsibilities of the Executive Board}
\renewcommand{\theenumi}{\arabic{enumi}} % For this section, we want items to use letters
\begin{enumerate}
	\item To hold a weekly meeting specific to their responsibilities and submit notes to CSH, if they are a Voting Member
	\item To meet at least weekly during the Standard Operating Session to discuss and report the operations of CSH
	\item To report pertinent information to members at the following House Meeting
	\item To maintain records of the goals defined by each previous E-Board
	\item To act as a Judicial Board as defined in \ref{Judicial}
	\item To review major projects as defined in \ref{Expectations of Active Members}
	\item To make the final vote regarding conditionals and appeals as defined in \ref{Membership Evaluation}
	\item To respect the privacy of members confiding in E-Board, barring situations related to sexual assault or endangerment of oneself or others
	\item To publish a document at the end of each semester to all members stating CSH's accomplishments of that semester
	      %\item To review and update the Constitution at the end of each Standard Operating Session, as defined in \ref{Standard Operating Session}.
	      %The constitution should remain up to date with current practices.
\end{enumerate}

\asubsubsection{Responsibilities of the Chairperson}
\begin{enumerate}
	\item To preside over E-Board and House Meetings
	\item To exercise general supervision over the operations of E-Board
	\item To exercise general supervision over regular CSH activities
	\item To act as a liaison to the academic and administrative departments at RIT
	\item To represent CSH publicly
	\item To coordinate collaboration between CSH and companies in the industry
	\item To cast the tie-breaking vote in a split decision in an E-Board Vote
\end{enumerate}

\asubsubsection{Responsibilities of the Evaluations Director}
\begin{enumerate}
	\item To exercise general supervision over Evals operations
	\item To oversee the screening, interviewing, and acceptance or rejection of prospective members
	\item To collaborate with the ResLife Advisor to determine room change selection and any changes of membership residency
	\item To vote on issues presented to E-Board
\end{enumerate}

\asubsubsection{Responsibilities of the Socials Director(s)}
\begin{enumerate}
	\item To exercise general supervision over Social operations
	\item To oversee the organization and execution of CSH social activities
	\item To ensure that there is a variety of events for members to participate in throughout the academic year
	\item To vote on issues presented to E-Board
\end{enumerate}

\asubsubsection{Responsibilities of the Public Relations Director}
\begin{enumerate}
	\item To operate CSH social media accounts intended to represent CSH as a whole
	\item To oversee the organization and execution of CSH philanthropic events
	\item To preserve and improve the public image of CSH
	\item To collaborate with other E-Board members to organize, execute, and advertise any events that are intended to be attended by persons who have never been Active Members
	\item To oversee the preparation of CSH for open houses, tours, and special events
	\item To vote on issues presented to E-Board
\end{enumerate}

\asubsubsection{Responsibilities of the Financial Director}
\begin{enumerate}
	\item To supervise financial administrators and transactions involving house projects
	\item To maintain financial and inventory records of CSH capital and assets
	\item To plan and enforce a CSH budget
	\item To oversee CSH finances and generation of CSH funds
	\item To publish a semesterly CSH financial statement
	\item To supervise the collection of semesterly dues
	\item To oversee the planning and execution of fundraising events
	\item To vote on issues presented to E-Board
\end{enumerate}

\asubsubsection{Responsibilities of the Research and Development Director(s)}
\begin{enumerate}
	\item To exercise general supervision over R\&D operations
	\item To oversee the planning, organization, and construction of technical projects for the CSH's benefit
	\item To strive to fulfill the CSH's need for technical equipment
	\item To organize and support Technical Seminars
	\item To vote on issues presented to E-Board
\end{enumerate}

\asubsubsection{Responsibilities of the House Improvements Director}
\begin{enumerate}
	\item To exercise general supervision over the IMPs operations
	\item To oversee the organization and construction of physical improvements to CSH
	\item To oversee the general maintenance of the appearance of CSH
	\item To vote on issues presented to E-Board
\end{enumerate}

\asubsubsection{Responsibilities of the Operational Communications Director}
\begin{enumerate}
	\item To represent the RTPs at E-Board Meetings and at House Meetings
	\item To report the status of the house systems, services, and networks
	\item To facilitate communications with RIT Information and Technology Services
	\item To vote on issues presented to E-Board
\end{enumerate}

\asubsubsection{Responsibilities of the History Director}
\begin{enumerate}
	\item To exercise general supervision over the History operations
	\item To maintain, uphold, and promote house traditions
	\item To collaborate with the Social Director(s) to ensure that all the Active, Alumni, and Advisory Members are informed of upcoming events
	\item To oversee the production and distribution of a semi-annual newsletter
	\item To vote on issues presented to E-Board
\end{enumerate}

\asubsubsection{Responsibilities of the Secretary}
\begin{enumerate}
	\item To ensure that minutes are recorded and posted for E-Board Meetings
	\item To ensure that minutes are recorded and posted for House Meetings
	\item To oversee the maintenance of E-Board records and documents
	\item To record all votes held and their outcomes during House Meetings and open E-Board Meetings
\end{enumerate}

\asubsubsection{Responsibilities of Ad Hoc Directorships}
\begin{enumerate}
	\item Ad Hoc Directorships are responsible for the task for which they were created
	\item Any responsibilities, budgets, or finances for the Ad Hoc Directorship are placed upon the assigned director
\end{enumerate}

\asubsection{Closed Executive Board}
Closed E-Board Meetings are open only to the Chair, Voting Members of E-Board, and those with the express permission of the aforementioned members.
A closed E-Board meeting may be called at any time by any member of E-Board.
However, the Chair and at least two-thirds of the Voting Members of E-Board must be present for the meeting to be called.

\asubsection{Qualifications}

\asubsubsection{Qualifications for Chairperson, Evaluations Director, Social Director, Financial Director, Research and Development Director(s), House Improvements Director, History Director, Public Relations Director}
\begin{enumerate}
	\item Candidates must be Active Members during the term of office.
	\item Elected or selected candidates may not hold two simultaneous voting E-Board positions, and must therefore resign their current position or decline a second position should they be elected or selected to a second voting position.
	\item Candidates for Chair or Evals Director must reside on floor during the term of office.
\end{enumerate}

\asubsubsection{Qualifications for Operational Communications Director}
\begin{enumerate}
	\item Candidates must be an RTP and an Active Member.
	\item An RTP cannot be the director if they currently hold any other elected E-Board position. % Consider that opcomm is also an exception before using this wording elsewhere
\end{enumerate}

\asubsubsection{Qualifications for Secretary or Ad Hoc Director}
\begin{enumerate}
	\item Candidates must be Active Members.
\end{enumerate}

\asubsubsection{Waiving of Qualifications}
\begin{enumerate}
	\item CSH may choose to waive the following subset of the qualifications for E-Board positions, as defined under \ref{Qualifications}, for all candidates:
	      \begin{enumerate}
		      \item Candidates must reside on floor during the term of office.
		      \item Social and R\&D are the only voting E-Board positions that allow for dual directorship.
	      \end{enumerate}
	\item When a waiver is proposed, the Chair of the Vote shall state which E-Board position and qualification is being voted upon, and any nominated candidates not qualified under \ref{Qualifications} who would become qualified if the waiver passes.
	\item A Three-Quarters Immediate Vote with three-quarters quorum is then taken to determine whether the proposed waiver shall take effect.
	\item Each vote to waive a qualification must only apply to a single qualification for a single E-Board position.
	\item Waivers always apply to all candidates for the E-Board position.
\end{enumerate}

\asubsection{Selection}

\asubsubsection{Dual Directors}
Social and R\&D are the only Voting E-Board positions that allow for dual directors.

If two candidates elect to run as dual directors, their names are placed together on a single line of the election ballot.
If they are also nominated as a single directorship or as dual directors with another member, their votes are not cumulative.
Each different nomination must be a separate entry on the election ballot.
Ad Hoc Directors are not restricted to single or dual directors.

The following special cases cover the operation of dual directors:
\begin{enumerate}
	\item If one of the members in a dual directorship resigns, or for any other reason ends the term of office, the other member in the dual directorship must also step down and the office becomes vacated.
	      The vacated office is then handled like any other vacated office in accordance with \ref{Executive Board Resignations}.
	\item During an official E-Board Vote, each dual director's vote counts for one half of a vote in the tallying of votes.
	      The members of the dual directorship need not vote the same way in a vote.
	      This cannot be used to give a vote to a position that would not otherwise have one in an E-Board Vote.
	\item A member of a dual directorship may not hold any other Voting E-Board position or the position of Chair.
	      When attendance requirements call for the dual directorship position to be present, at least one of the dual directors must be present to fulfill the requirements.
\end{enumerate}

\asubsubsection{Selection of the Chairperson, Evaluations Director, Social Director(s), Financial Director, House Improvements Director, History Director,  Research and Development Director(s), Public Relations Director}
\begin{enumerate}
	\item The opening of the E-Board position(s) is announced at a House Meeting and nominations for the position are taken for a minimum of a seventy-two hour period.
	      Any member may nominate any member, or pair of members where appropriate, for a directorship.
	\item The candidates will be notified of their nomination.
	      Each candidate is given a minimum of a twenty-four hour period to accept or decline the nomination.
	      A list of all nominees who have accepted their nominations will be posted shortly thereafter.
	\item The date and time of candidate speeches will be announced by a member of E-Board at least five days prior to the event.
	      All Active Members are expected to attend candidate speeches in accordance with \ref{Expectations of Active Members}.
	      If a member cannot attend due to a scheduling conflict, they may provide a reason for their absence to the Evals Director, who will record the reason with the absence.
	\item Each candidate will be given an equal amount of time to present their platform of candidacy to the attending members.
	\item Ballots will then be distributed for a Balloted Vote \ref{Balloted Vote} and voting will be open for a minimum of a forty-eight hour period.
	      The Ballots will list, in random order, all of the candidates who are qualified for a given office with a means to indicate the selection of one of the candidates.
	      In addition, an area will be provided to indicate a write-in selection of a candidate.
	\item At the end of the voting period, the Chair will terminate voting.
	\item The winners are determined via the process described in \ref{Ranked Choice}.
	      A fifty-percent quorum is required for the election to be official.
	      All winners are notified of their election.
	      If the position is currently vacant, the winner immediately assumes office.
	      If not, the winner will assume office at the end of the current term, as defined in \ref{Term}.
	      Any office whose winner declines the election, or whose winner does not fulfill the requirements of the elected office, shall have their votes redistributed in the same process as a loser in \ref{Ranked Choice}.
\end{enumerate}

\asubsubsection{Selection of the Operational Communications Director}
\renewcommand{\theenumi}{\alph{enumi}} % For this section, we want items to use letters

\begin{enumerate}
	\item A candidate for the OpComm Directorship shall be chosen by the current RTPs.
	\item The candidate is given a minimum of twenty-four hours to accept or decline the nomination.
	\item The candidate is presented to E-Board for approval.
	      This E-Board Meeting is closed to the E-Board Members, RTPs, and members with explicit invitation from E-Board.
	\item All Voting E-Board Members must be present during the discussion and voting period unless that member is a candidate for the position, in which case the member is absent and their vote is abstained.
	      An E-Board Vote is taken to determine whether the candidate is selected for the position.
\end{enumerate}

\asubsubsection{Selection of Ad Hoc Directors}
\begin{enumerate}
	\item When a group of members feels an Ad Hoc Directorship is necessary, they present their plans to E-Board.
	      An E-Board Vote is taken to determine whether directorship status is granted.
	\item When the Ad Hoc Directorship is granted status, a director is appointed, and duties, budget, and membership considerations are defined.
\end{enumerate}

\asubsubsection{Selection of the Secretary}
E-Board may select any interested Active Member as Secretary.
The selection process can be an informal appointment, or follow an election process similar to other Voting E-Board positions.

\asubsection{Executive Board Resignations}
An E-Board Member may resign their position by submitting in writing the reason for resignation to the Chair.
The E-Board resignation will be read at the following House Meeting and become effective at that time.
The office will become vacant and the selection process for a new director, as described in \ref{Appointment of an Interim Director}, will begin at that time.
The selection process, as defined in \ref{Selection}, of a replacement for the position vacated must begin within two weeks from when the resignation took place.
To postpone such a selection, the Chair may chair an Immediate Relative Majority Vote during a House Meeting prior to the beginning of the selection process to delay the selection process by a specified amount of time.

\asubsection{Selection of an Acting Chairperson}
If there is not an acting Chair during a process where one is required, then the following line of succession is used to determine which director becomes the acting Chair until a new Chair is elected.
\begin{enumerate}
	\item[1.] Evals Director
	\item[2.] Financials Director
	\item[3.] Voting E-Board Member chosen by an E-Board Vote
\end{enumerate}

\asubsection{Appointment of an Interim Director}
The duties and responsibilities of a vacated office are assumed by an Active Member that is appointed at the Chair's discretion until the new director takes office. % It is often the Chair that assumes the responsibilities of that position
The vote of a vacated E-Board position shall be cast as an abstention in all E-Board matters where this vote is required to be cast.

\asubsubsection{Impeachment}
\begin{enumerate}
	\item Impeachment of any E-Board Member may be initiated by petition, in writing, consisting of a minimum number of signatures of current members equaling or exceeding one-third the number of Eligible Votes, as defined in \ref{Eligible Votes}.
	\item The impeachment petition is then presented at an E-Board Meeting.
	      The member(s) initiating the petition present their case to E-Board.
	      E-Board then questions the accused member of the allegations.
	\item An E-Board Vote is taken on the impeachment petition, with all Voting E-Board Members present except the accused member who must be absent and whose vote counts as an Abstention, to determine if the allegations stated in the petition are legitimate grounds for impeachment.
	      If the majority of E-Board votes are negative, the petition and impeachment proceedings are dismissed.
	      This vote may be overridden by an impeachment petition consisting of the grounds for impeachment and a minimum number of signatures of current members equaling or exceeding two-thirds the number of Eligible Votes.
	\item If the majority of the E-Board votes are positive, or the negative vote was overridden, the petition is presented at the following House Meeting and the accused and accuser(s) again present their cases.
	\item Ballots will then be distributed and a Secret Ballot Vote shall be held for a minimum of a forty-eight hour period to determine whether or not the member should be removed from office.
	      Votes shall be collected and counted for a Two-Thirds Balloted Vote, as described in \ref{Balloted Vote}.
	\item A two-thirds quorum is necessary for the vote to be official.
	      If the resolution passes, the accused officer is relinquished of their position and any benefits thereof and this position is treated like any other vacated position.
	      If a quorum cannot be reached after two attempts, or the percentage of affirmative votes does not equal or exceed the minimum, impeachment proceedings are dismissed.
	      A new selection and interim duty fulfillment procedure is followed similar to that of a resignation, as described in \ref{Executive Board Resignations}.
	\item The Secretary can be impeached according to the above process, or may be dismissed by an E-Board Vote.
\end{enumerate}

\asubsubsection{Term}
\begin{enumerate}
	\item The election process for the following year's E-Board members shall begin in the middle of spring semester.
	\item The newly selected officers shall begin their terms on June 1 of that year and their terms shall end on May 31 of the following year.
	\item The term of an officer will be abbreviated due to resignation, impeachment, or change in membership status.
	\item Officers elected or selected during the course of the year due to an abbreviated term of the previous officer shall hold office until the end of the normal term.
	\item When the task of an Ad Hoc Directorship has been completed, the directorship dissolves.
	      When an Ad Hoc Director resigns, the directorship dissolves and must be reinstated with a new director.
\end{enumerate}

\asubsection{Appeals}
To initiate an appeal, a member must have the support of three Voting E-Board Members, or a petition with the signatures of one-third of Active Members.
After the appeal is presented, an E-Board Vote is taken to determine whether or not to overturn and reevaluate the decision.
If the vote passes, E-Board may discuss and make a new ruling by another E-Board Vote.

\asection{Constitutional Maintainers}

\asubsection{Constitutional Maintainer Qualifications}
Maintainers must be Active or Alumni Members.

\asubsection{Constitutional Maintainer Expectations}
Maintainers are expected to:
\begin{enumerate}
	\item Be knowledgeable about the Constitution
	\item Review changes to the Constitution for grammar, spelling, and internal consistency
	\item Participate in discussion of proposals and amendments
	\item Keep a public record of changes to the Constitution
\end{enumerate}
Failure to meet any of these expectations is grounds for revocation of Maintainer status by E-Board.

\asubsection{Constitutional Maintainer Term}
Maintainer status lasts until it is resigned, or until it is revoked by an E-Board Vote.
A Maintainer may resign at any time by notifying the current E-Board and Maintainer group.
If a Maintainer no longer satisfies \ref{Constitutional Maintainer Qualifications}, they lose Maintainer status.

\asubsection{Constitutional Maintainer Selection}
Any member may nominate a qualified member for Maintainer status to E-Board for consideration.
E-Board may choose to approve or reject the nomination by E-Board Vote with a quorum of seventy-five percent of the Eligible Votes.

\asection{Root Type Persons}
The OpComm Directorship is responsible for overseeing the implementation of maintenance and upgrades to the CSH computer systems networks.
It is a self-governing committee making decisions by Immediate Relative Majority vote with two-thirds quorum of current RTPs.
Membership is composed of all RTPs.

\asubsection{Selection of a Root Type Person}
\renewcommand{\theenumi}{\alph{enumi}} % For this section, we want items to use letters
\begin{enumerate}
	\item Nominations are taken from the RTPs meeting the selection criteria in \ref{Qualifications of a Root Type Person}.
	\item Each candidate is given a minimum of twenty-four hour period to accept or decline the nomination.
	\item A list of all nominees who have accepted is presented to E-Board for approval.
	      This E-Board Meeting is closed to E-Board Members, RTPs, and members with explicit invitation from E-Board.
	\item If an E-Board Member is a candidate for the office in discussion, the member is absent and their vote is abstained.
	      An E-Board Vote, as described in \ref{Executive Board Vote}, is taken to determine whether the nominations of the RTP(s) are accepted.
\end{enumerate}

\asubsubsection{Qualifications of a Root Type Person}
Candidates must be Active Members.

\asubsubsection{Prior Root Type Persons}
Prior RTPs are those members who are no longer current Active Members and have not been granted an extension by the current RTPs.
Prior RTPs are not guaranteed access to the current root passwords and other authentication tokens.
The current RTPs may draft rules and regulations specifying the rights and privileges of Prior RTPs.

\asubsubsection{Creation of Accounts}
RTPs have the authority to manage user accounts for CSH systems.
Before a member may receive an account they must:
\renewcommand{\theenumi}{\arabic{enumi}} % For this section, we want items to use letters
\begin{enumerate}
	\item Sign the Code of Conduct sheets pertaining to the responsible utilization of CSH and RIT facilities.
	\item Obtain greater than or equal to 60\% of required signatures, excluding those of On-Floor Members who have not passed a Membership Eval, in the Intro Packet or successfully complete Intro Eval.
	\item Sign a copy of the Membership Agreement
\end{enumerate}
Accounts for Honorary and Advisory Members may be created at the discretion of the RTPs.

\asubsubsection{Code of Conduct}
The Code of Conduct located at \url{https://github.com/ComputerScienceHouse/CodeOfConduct} is the canonical Code of Conduct for CSH accounts.
\\* \\*
Members are bound to the Code of Conduct revision that they sign when initially creating their account.
Members may sign a more recent revision of the Code of Conduct to update their agreement.

\asubsubsubsection{Changes}
\renewcommand{\theenumi}{\alph{enumi}} % For this section, we want items to use letters
The following process is used for making changes to the Code of Conduct document.
\begin{enumerate}
	\item An RTP drafts a change to the Code of Conduct and makes publicly available both the summary and difference file of the change.
	\item The change is proposed at a House Meeting and is discussed at the same House Meeting.
	\item The final proposal is presented at the next House Meeting and ballots are distributed for a One-Half Balloted House Vote with fifty percent quorum, as described in \ref{Balloted Vote}.
	\item Any RTP may make any non-semantic change to the Code of Conduct with the approval of two other RTPs.
\end{enumerate}

% ARTICLE VI - GENERAL OPERATIONS OF THE HOUSE
\article{General Operations of the House}

\asection{Financial Operations of the House}

\asubsection{Amount of House Dues}
The amount of dues for Active Members will be eighty dollars per academic semester.

\asubsection{Collection of Dues}
The collection period of house dues will be decided in conjunction with the Center for ResLife and Housing Operations.
During the dues collection period, dues for both semesters are collected through the member’s RIT bill.
Dues for any Alumni in good standing are to be collected when intention to pay is expressed to the Financial Director.

\asubsubsection{Rules for Giving Exceptions and Exemptions for House Dues}
Members may appeal their situation to the Financial Director or E-board in the case that the member is dissatisfied with the Financial Director's decision.
The Financial Director, or E-board, then decides whether their situation warrants an extension to the payment deadline and/or reduction of the required dues.
\\*\\*
If the appeal is denied, the member’s payment is considered delinquent and the member’s privileges are revoked until payment can be made.
All dues must be collected in full before the annual Membership Eval, as defined in \ref{Membership Evaluation}.
After this date, dues collection is suspended until the start of the next academic year.
\asubsection{Breakdown of Dues for Directorship Budgets}
\begin{center}
	\begin{tabular}[c]{|l c|}
		\hline
		Directorship & Percentage of Dues
		\\ \hline
		\hline
		OpComm       & 20\%
		\\ \hline
		Social       & 20\%
		\\ \hline
		R\&D         & 20\%
		\\ \hline
		IMPs         & 15\%
		\\ \hline
		History      & 10\%
		\\ \hline
		Evals        & 5\%
		\\ \hline
		PR           & 5\%
		\\ \hline
		Accumulated  & 5\%
		\\ \hline
	\end{tabular}
\end{center}

% The suggested total operating budget is $8360.
% This figure comes from having approximately 55 on-floor members with $80 per semester dues (totaling $8800) minus 5% set aside for Accumulated.
Dues collected from on-floor members are to be distributed into directorship budgets as shown in the above table.
Money allocated for Accumulated or collected from off-floor members is deposited into the CSH account and saved.

\asubsection{Expenditure Approval}
All CSH expenditures must be approved by the Financial Director and the director whose budget is going to be used for the expenditure.
Single expenditures may not exceed seventy-five dollars and Total Expenditures may not exceed one-hundred dollars.
\\* \\*
If the above amounts are to be exceeded, the expenditure must be approved by an Immediate One-Half Vote at a House Meeting before the funds may be appropriated.
At the Financial Director's discretion, a Spending Committee meeting may be held in order to approve any purchase exceeding one-hundred dollars, but not to exceed three-hundred dollars.
A Two-Thirds Immediate Vote with twenty-five percent quorum is required to approve the purchase.
\\*\\*
For expenditures exceeding \$300 in total funding, a spending presentation must be presented at a House Meeting.
This presentation includes the funds required for the expenditure, inventory of required resources, and, if applicable, a timeline for completion.
\\* \\*
If an appropriate directorship cannot be determined for an expenditure, it is to be brought up for approval at a House Meeting as a miscellaneous expenditure and approved by an Immediate Relative Majority vote with fifty percent quorum, regardless of the amount.
This amount is to be directly subtracted from the general CSH account.
% HOUSING
\asection{Housing}

\asubsection{Housing Status}
All Alumni, Active, and Intro Members have a Housing Status.
This status indicates their privilege of priority housing on the floor.
Alumni Members in good standing keep the housing status they had as Active Members.
Alumni Members in bad standing have Off-Floor Status.

\asubsubsection{On-Floor Status}
Members with On-Floor Status are eligible to live on the floor.
To be granted On-Floor Status, members may notify the Evals Director that they would like to come up for a vote.
The Evals Director will then bring them up for vote at the next Evals Meeting.

\asubsubsection{Off-Floor Status}
Members with Off-Floor Status do not have the right to live on the floor.
They still have access to all other privileges associated with their membership, and may still accumulate Housing Priority Points.

\asubsection{Housing Priority System}
The Housing Priority System is a means for determining the priority a member has in a housing issue or the assignment of available housing.
The member with top priority is the member with On-Floor Status and the most Housing Priority Points.
Housing Priority Points are accumulated once per Operating Session at the conclusion of the Membership Eval.
Each Active Member who passes the Membership Eval is granted two Housing Priority Points.

\asection{Evaluations Processes}
At the beginning of any Eval Process, the Evals Director must read the sections of the CSH Constitution regarding the relevant Eval Process.
It is incumbent upon each member to provide the Evals Director with whatever information they feel is necessary to ensure an accurate evaluation.

\asubsection{Appeals Process}
If a member disagrees with the outcome of any non-passing Eval and wishes to appeal the decision, they may do so as stated in \ref{Appeals}.
\\* \\*
If the member is still unsatisfied after being heard by E-Board, the appeal may be brought to the attention of the ResLife Advisor.

% ARTICLE VII - VOTING
\article{Voting}
This section outlines the different types of votes and ballots used to make decisions and defines relevant terminology.

\asection{Definitions}

\asubsection{Eligible Votes}
The number of Active Members eligible to vote on the issue.

\asubsection{Votes Cast}
The number of ballots cast in a vote, excluding abstentions.

\asubsection{Quorum}
The minimum number of votes that must be cast for the vote to be considered valid.
Any member present for an Immediate Vote or given a ballot who does not explicitly cast their vote is counted as an abstention.
A quorum is expressed as a fraction or percentage of Eligible Votes, rounded up to the nearest vote, unless otherwise specified in the text of the vote.

\asubsection{Proxy Vote}
A vote cast by one member on behalf of another.
Proxy Votes are only permissible at the discretion of the Chair of the Vote, and may be disallowed in the text describing the vote.
Proxy Votes must be documented in writing and signed by the absent member.
The count of all Proxy Votes must be recorded and announced in all votes.

\asubsection{Abstention}
An abstention is a neutral vote that counts towards the quorum but not towards determining the outcome of a vote.
A means to abstain must always be provided in a vote.

\asubsection{Vote Counters}
The Chair of the Vote is a vote counter and will additionally select two other members to count votes.

\asection{Voting Procedures}

\asubsection{Balloted Vote}

\asubsubsection{Method of Vote}
Votes are cast on ballots that provide a means to indicate every possible option in the vote.

\asubsubsection{Voting Period}
For constitutional modification, candidate selection, and officer removal votes, the voting period must be at least forty-eight hours in length.
For any other type of vote, the voting period must be at least twenty-four hours.
The minimum length of the voting period may be explicitly lengthened, but never shortened, in the text describing the actual vote.

The voting period opens once ballots are distributed to each member eligible to cast a vote.
At the end of the voting period, the Chair of the Vote collects the ballots, closing the voting period.
The Vote Counters then tally the results.

\asubsection{Immediate Vote}

\asubsubsection{Method of Vote}
The Chair of the Vote will state all possible ways to vote, then call out each possibility one at a time.
The chairing member will count the number of members casting their immediate vote for that possibility.
Any members that count towards quorum that do not explicitly cast a vote will have their vote counted as an abstention.

\asubsubsection{Voting Period}
An Immediate Vote lasts as long as it takes for all votes to be tallied.

\asubsubsection{Secret Immediate Vote}
An Immediate Vote in which votes are kept anonymous.

\asubsubsection{Executive Board Vote}
In an E-Board Vote, only Voting E-Board Members are eligible to vote.
Unless otherwise specified, E-Board votes are Immediate Relative Majority Votes with a fifty percent quorum.

\asubsection{Batch Vote}
When a Batch Vote is called for by the Chair of the Vote, a non-empty subset of the voting docket may be amended to a single vote.
A Two-Thirds Immediate Vote is required to allow a Batch Vote to take place.
If the call for Batch Vote passes, the subset may then be voted on; otherwise, the docket remains unchanged.

\asection{Approval Criteria}
Approval Criteria are the rules that determine whether an option in a vote has passed.

\asubsection{Relative Majority}
In a Relative Majority Vote, an option in the vote passes if the number of votes cast for that option is larger than the number of votes cast for every other option individually.

\asubsection{Fractional}
In a Fractional Vote, an option in the vote passes if the number of votes cast for that option exceeds a specified fraction of the Votes Cast, rounded up to the nearest vote.
The fraction is specified in the text describing the vote, for example, "Two-Thirds Vote".

\asubsection{Ranked Choice}
In a Ranked Choice Vote, voters rank the options by placing a '1' by their first choice, a '2' by their second choice, and so on, until they no longer wish to express any further preferences or run out of options.

The winning option is selected outright if it gains more than half the votes cast as a first preference.
If not, the option with the fewest number of first preference votes is eliminated and their votes move to the second preference marked on the ballots.
This process continues until one option has half of the votes cast and is elected.

\asection{Ties Between Vote Options}

\asubsection{With Pass/Fail}
If the number of votes cast for the pass option equals the number of votes cast for the fail option, then the vote has failed.

\asubsection{With Multiple Options}
If only one option may pass, then the vote must be recast or tabled at the discretion of the Chair of the Vote.
In the event of a tie in an E-Board Vote, the Chair may cast the tie-breaking vote.

% ARTICLE VIII - JUDICIAL
\article{Judicial}
E-Board may approve a request from a member for a Judicial proceeding when an official clarification of the Constitution is required, there is a conflict among members, or a House interest needs resolution.

\asection{Formation of a Judicial Board}
A Judicial Board is made up of the Chair, the Evals Director, and an additional Voting E-Board Member.
If the Chair or the Evals Director are deemed biased or unfit for a position on the Judicial Board, they will be replaced by another Voting E-Board Member by means of an E-Board Vote of the remaining members.

\asection{Judicial Investigation}
The Judicial Board will be responsible for making all necessary inquiries into the matter of the request brought to the Judicial Board.

\asection{Judicial Ruling}
The Judicial Board may present an official ruling following the Judicial Investigation.
This ruling may be appealed in the same manner as an E-Board decision, as defined in \ref{Appeals}.

% ARTICLE IX - ADHERENCE TO UNIVERSITY POLICIES
\article{ADHERENCE TO UNIVERSITY POLICIES}

\asection{Anti-Hazing}

\asubsection{RIT Hazing Policy (RIT Student Conduct Process; IV. RIT Code of Conduct; 14. Hazing/Failure to Report Hazing)}
Hazing/Failure to Report Hazing.
Behavior, regardless of intent, which endangers the emotional, or physical health and safety of a Student for the purpose of membership, affiliation with, or maintaining membership in, a group or Student Organization.
Hazing includes any level of participation, such as being in the presence, having awareness of hazing, or failing to report hazing.
Examples of hazing include, but are not limited to, beating or branding, sleep deprivation or causing excessive fatigue, threats of harm, forcing or coercing consumption of food, water, alcohol or other drugs, or other substances, verbal abuse, embarrassing, humiliating, or degrading acts, or activities that induce, cause or require the Student to perform a duty or task which is not consistent with fraternal law, ritual or policy or involves a violation of local, state or federal laws, or the RIT Code of Conduct.

\asubsection{NY State Hazing Law}

\asubsubsection{§ 120.16 Hazing in the first degree.}
A person is guilty of hazing in the first degree when, in the course of another person's initiation into or affiliation with any organization, he intentionally or recklessly engages in conduct which creates a substantial risk of physical injury to such other person or a third person and thereby causes such injury.
Hazing in the first degree is a class A misdemeanor.

\asubsubsection{§ 120.17 Hazing in the second degree.}
A person is guilty of hazing in the second degree when, in the course of another person's initiation or affiliation with any organization, he intentionally or recklessly engages in conduct which creates a substantial risk of physical injury to such other person or a third person.
Hazing in the second degree is a violation.

\asection{Anti-Discrimination Clause}
This organization shall not discriminate on the basis of sex, race, color, sexual orientation, gender identity and gender expression, religion, age marital state, national origin, disability or veteran status.
This policy will include but is not limited to, recruiting, membership, organization activities or opportunities to hold or run for club office.

\asection{Statement of Compliance with University Regulations}
This organization shall comply with all university and Center for Campus Life policies and regulations, and local, state and federal laws.

\end{document}
