\documentclass{article}
\usepackage{hyperref}
\usepackage{showkeys}
\usepackage{enumerate}

% Title page information
\title{
\textbf{Computer Science House}\\
Membership Agreement and Code of Conduct}
\author{}

% Last Modified Date
\newcommand{\datechanged}{Last Updated: \today}
\date{\datechanged}

% Fix margins
\setlength{\evensidemargin}{0in}
\setlength{\oddsidemargin}{0in}
\setlength{\textwidth}{6.5in}
\setlength{\topmargin}{0in}
\setlength{\textheight}{8.5in}

\pagestyle{myheadings}
\markright{{\rm CSH Membership Agreement \hfill \datechanged \hfill Page }}

\begin{document}
\maketitle

\section*{Our Pledge}

In the interest of fostering an open and welcoming environment, we as
members of Computer Science House pledge to make membership and
participation in our community a harassment-free experience for everyone,
regardless of age, body size, disability, ethnicity, sex characteristics,
gender identity and expression, level of experience, education,
socio-economic status, nationality, personal appearance, race, religion,
or sexual identity and orientation. In the interest of being good
citizens in the wider RIT community, we also pledge to abide by the
\href{https://www.rit.edu/studentaffairs/studentconduct/code-conduct}
{RIT Student Code of Conduct} and uphold its values.

\section*{Our Standards}

Examples of behavior that contributes to creating a positive environment
include:

\begin{itemize}
    \item Using welcoming and inclusive language
    \item Being respectful of differing viewpoints and experiences
    \item Gracefully accepting constructive criticism
    \item Focusing on what is best for the community
    \item Showing empathy towards other community members
\end{itemize}

\begin{flushleft}
Examples of unacceptable behavior by participants include (but are not
limited to):
\end{flushleft}

\begin{itemize}
    \item The use of sexualized language or imagery and unwelcome sexual
          attention or advances
    \item Inflammatory behavior, insulting/derogatory comments, and personal
          or political attacks
    \item Public or private harassment
    \item Publishing others’ private information, such as a physical or
          electronic address, without explicit permission
\end{itemize}

\section*{Our Responsibilities}

Computer Science House’s Executive Board (E-board) is responsible for
clarifying the standards of acceptable behavior and are expected to take
appropriate and fair corrective action in response to any instances of
unacceptable behavior. E-Board has the right and responsibility to remove,
edit, or reject posts, comments, or other media that is not aligned to this
Code of Conduct from official CSH communication or collaboration platforms,
or to ban temporarily or permanently any member from these platforms for
other behaviors that they deem inappropriate, threatening, offensive, or
harmful.

\section*{Scope}

This Code of Conduct applies within all CSH spaces, and it also applies when
an individual is representing CSH or its community in public spaces. Examples
of representing CSH include using an official CSH e-mail address, posting via
an official social media account, or acting as an appointed representative at
an online or offline event. Representation of CSH may be further defined and
clarified by E-Board.

\section*{Enforcement}

Instances of abusive, harassing, or otherwise unacceptable behavior may be
reported by contacting CSH E-board at
\href{mailto:eboard@csh.rit.edu}{eboard@csh.rit.edu}. All complaints will
be reviewed and investigated and will result in a response that is deemed
necessary and appropriate to the circumstances. E-board is expected to
reasonably protect the privacy of the reporter of an incident, but may
escalate the report to res-life or other RIT faculty at their discretion.
Further details of specific enforcement policies may be posted separately.
Members who do not follow or enforce the Code of Conduct in good faith may
face temporary or permanent repercussions as determined by E-Board.

\section*{Attribution}

This Code of Conduct is adapted from the
\href{https://www.contributor-covenant.org/}{Contributor Covenant},
version 1.4, available at \\
\href{https://www.contributor-covenant.org/version/1/4/code-of-conduct.html}
{https://www.contributor-covenant.org/version/1/4/code-of-conduct.html} \\
For answers to common questions about this code of conduct, see
\href{https://www.contributor-covenant.org/faq}
{https://www.contributor-covenant.org/faq}.

\section*{Commitment}

I, \makebox[2.5in]{\hrulefill}, hereby agree to the following conditions for
membership in Computer Science House:

\begin{enumerate}
    \item I will abide by the qualifications and expectations of a member,
          as well as follow the general rules of the House, as outlined in the CSH
          Constitution for the entirety of my membership.
    \item I commit to making Computer Science House a friendly and welcoming
          community by embracing the values set forth in this Code of Conduct.
    \item I will abide by the guidelines and rules set forth in the CSH
          Computer Code of Conduct regarding CSH services and member privileges for
          the entirety of my membership.
    \item If I am to fail in the evaluations process, I understand that I will be
          removed from any housing queue or board. If I am already registered to live
          on CSH the following year, I understand my spot will not be revoked.
\end{enumerate}

I understand that should I violate any of the above conditions, I may be
subject to sanctions as outlined in the CSH Constitution.

\noindent
\begin{tabular}{ll}
\\[8ex]
\makebox[3.5in]{\hrulefill} & \makebox[2.5in]{\hrulefill}\\
Signature & Date\\
\end{tabular}

\end{document}
